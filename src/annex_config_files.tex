\section{Annexes}

\subsection{Configuration file: static\_config.ini}
\label{annex:staticconfig}

The \emph{static configuration file} is the file dealing with the parameters
that user will preferably not change. It includes the communication 
settings with the Yocto-Pictor and the server
credentials. As it contains critical settings, this file will not be 
automaticaly synchronized over the network.

\subsubsection{Section: yoctopuce}
\label{annex:configYocto}
\begin{tabularx}{\textwidth} {
        | >{\hsize=.15\hsize}X
        | >{\hsize=.6\hsize}X
		| >{\hsize=.25\hsize}X | }
	\hline
	\textbf{field} & \textbf{definition} & values (\textbf{default}) \\
	\hline
	\hline
	yoctopuce\_ip &	Yocto-Pictor-Wifi IP Address & 10.42.0.X
	\\ \hline
	yocto\_prefix1 & Yocto-Pictor Serial (relay board) & OBSVLFR1-XXXXXX
	\\ \hline
	yocto\_prefix2 & Yocto-Pictor-Wifi Serial &	OBSVLFR2-XXXXXX
	\\ \hline
	yocto\_gps & Yocto-GPS-V2 Serial & YGNSSMK2-XXXXXX
	\\ \hline
	bypass\_yocto & Bypass Yocto-Pictor during in a sequence (debug mode) & 
	\textbf{no} \\ \hline
\end{tabularx}

\subsubsection{Section: network}
\label{annex:configNetwork}

\begin{tabularx}{\textwidth} {
		| >{\hsize=.2\hsize}X
        | >{\hsize=.65\hsize}X
		| >{\hsize=.15\hsize}X | }
	\hline
	\textbf{field} & \textbf{definition} & \textbf{default} \\
	\hline
	\hline
	credentials & user account and server name & user@server 
	\\ \hline
	remote\_dir & name of the remote directory where data/config/logs are
	pushed & \texttildelow/XXYY 
	\\ \hline

	ssh\_port &	ssh port used to connect to the server & 22
	\\ \hline

	remote\_ssh\_port & ssh port used to connect from the server to the host
	system (use ssh -p 20213 rugged\_pc\_user@127.0.0.1 to connect)
	& 20213
	\\ \hline

\end{tabularx}

	%  webcam\_sky
	%  webcam\_site & credentials (user/pass) + IP address of the webcams & 
	%  USER:PASS\_1@ ip\_cam\_sky
	%  USER:PASS\_2@ ip\_cam\_site  & 
	%  see section \textcolor{red}{TODO} 
	%  & \\ \hline


\subsection{Configuration file : dynamic\_config.ini}
\label{annex:dynamicconfig}
The \emph{dynamic configuration file} is a communication tool allowing the
user to remotely change settings of the system. Unlike the \emph{static
configuration file}, it is synchronized with an identical file on the server side.
Synchronization over the network is performed using the one with the more recent 
modification.

\subsubsection{Section: general}
\begin{tabularx}{\textwidth} {
        | >{\hsize=.15\hsize}X
        | >{\hsize=.65\hsize}X
		| >{\hsize=.2\hsize}X | }
	\hline
	\textbf{field} & \textbf{definition} & values (\textbf{default}) \\
	\hline
	\hline
	keep\_pc & state of the system at the end of the sequence & on / \textbf{off} 
	\\ \hline

	start\_sequence & start or no the sequence every time the system is turned 
	on & \textbf{yes} / no 
	\\ \hline

	auto\_update & check for update at start-up & yes / \textbf{no}
	\\ \hline

	sequence\_file 	& which protocol file to follow & \textit{path to file}
	\\ \hline
\end{tabularx}

\subsubsection{Section: GPS}
\begin{tabularx}{\textwidth} {
        | >{\hsize=.15\hsize}X
		| >{\hsize=.85\hsize}X| }
	\hline
	\textbf{field} & \textbf{definition} \\
	\hline
	\hline
	latitude  &  Cartesian latitude of the location (float)
	\\ \hline
	longitude & Cartesian longitude of the location (float)
	\\ \hline
\end{tabularx}


\subsubsection{Section: pantilt}
\begin{tabularx}{\textwidth} {
        | >{\hsize=.15\hsize}X
        | >{\hsize=.6\hsize}X
		| >{\hsize=.25\hsize}X | }
	\hline
	\textbf{field} & \textbf{definition} & values (\textbf{default}) \\
	\hline
	\hline
	offset\_pan    & Azimuth adjustement value (in degrees) to point to South or North & 
	[0 - 360] \textbf{0} \\ \hline
	offset\_tilt   & Zenith adjustement value (in degrees) to point to point
	the Nadir &
	[0 - 360] \textbf{60} \\ \hline
	reverse\_tilt & Change the sign of both Azimuth and Zenith &
	yes / \textbf{no} \\
\hline
\end{tabularx}


\subsubsection{Section: metadata}
Metadatas are the information embedded in each Hypernets Sequence folder. 
In particular, each metadata file will have \emph{a header}, that is, 
the top the file. This header contains some dynamic informations like 
the current date and time as well as static ones, for instance the site
identification. This section of the \emph{dynamic config file} is the 
exact content of each metadata header. Fields between brackets will be
automatically processed by the host system.

\begin{table}[H]
	\centering
	\begin{tabularx}{\textwidth} {
	        | >{\hsize=.25\hsize}X
	        | >{\hsize=.50\hsize}X
			| >{\hsize=.25\hsize}X | }
		\hline
		\textbf{field} & \textbf{definition} & values  \\
		\hline
		\hline

		principal\_invesigator & Name of the PI, & 
		\textit{Investigator name} 
		\\ \hline

		datetime & UTC date and time at the sequence start time (no need to edit) 
		& \{datetime\} \\ \hline

		hypstar\_sn &  Hypstar Serial Number  & 123456\\ \hline
		site\_id & A site ID with the naming convention: XXYY such as: \newline
			-- XX: two first letters of the site name  \newline
			-- YY two first letters of the country partners-name 
		& XXYY \\ \hline

		protocol\_file\_name & Self-reference to the \textit{protocol file name} 
		from the \textit{general section} of this file (no need to edit) &  
		\$\{general:sequence\_file\} 
		\\\hline

		latitude & Self-reference to the latitude from the GPS section of this file 
		(no need to edit) & \$\{GPS:latitude\} \\ \hline

		longitude & Self-reference to the longitude from the GPS section of this file 
		(no need to edit) & \$\{GPS:longitude\}\\ \hline

		offset\_pan\ & Self-reference to the \textit{protocol file name} 
		from the \textit{pantilt} of this file (no need to edit) &  
		\$\{pantilt:offset\_pan\} 
		\\\hline

		offset\_tilt\ & Self-reference to the \textit{protocol file name} 
		from the \textit{pantilt} of this file (no need to edit) &  
		\$\{pantilt:offset\_tilt\} 
		\\\hline

		offset\_tilt\ & Self-reference to the \textit{protocol file name} 
		from the \textit{pantilt} of this file (no need to edit) &  
		\$\{pantilt:offset\_tilt\} 
		\\\hline

		azimuth\_switch\ & Self-reference to the \textit{protocol file name} 
		from the \textit{pantilt} of this file (no need to edit) &  
		\$\{pantilt:azimuth\_switch\} 
		\\\hline

	\end{tabularx}
\end{table}
\noindent\textbf{Note :} if you want to see what your metadata header will look like,
you can use this command :
\vspace{-10pt}
\begin{lstlisting}
python -m hypernets.abstract.create_metadata
\end{lstlisting}

\subsubsection{Section: hypstar}

\begin{table}[H]
	\centering
	\begin{tabularx}{\textwidth} {
			| >{\hsize=.15\hsize}X
			| >{\hsize=.55\hsize}X
			| >{\hsize=.30\hsize}X | }
		\hline
		\textbf{field} & \textbf{definition} & values (\textbf{default}) \\
		\hline
		\hline
		hypstar\_port & path to the hypstar \textit{device file} 
		& \textbf{/dev/radiometer0} \newline
		/dev/ttyUSB5 
		\\ \hline
		baudrate & speed communication (in bauds) with the instrument 
		& \textbf{115200}, 460800, 921600, 3000000, 6000000, 8000000
		\\ \hline
		loglevel & Set the log verbosity of the communication with the instrument 
		& \textbf{ERROR}, INFO, DEBUG, TRACE 
		\\ \hline
		swir\_tec & Thermoelectric Cooler setting point (in degree Celsius) 
		the SWIR module
		& [ -15 ; +40 ] \textbf{0}
		\\ \hline
	\end{tabularx}
\end{table}


\begin{table}[H]
	\centering
	\begin{tabularx}{\textwidth} {
			| >{\hsize=.2\hsize}X
			| >{\hsize=.8\hsize}X|}
		\hline
		\textbf{Log Level} & \textbf{Definition}  \\
		\hline
		\hline
		ERROR & only errors are reported on stderr \\ \hline
		INFO & stdout + stderr \\ \hline
		DEBUG & driver command execution printout to stdout \\ \hline
		TRACE & low level communication bytes are printed to stdout \\
		\hline
	\end{tabularx}
\end{table}

%\begin{sidewaystable}
% Webcams: see this
% \href{https://github.com/HYPERNETS/hypernets\_tools/issues/35}{github issue}.
%\end{sidewaystable}
