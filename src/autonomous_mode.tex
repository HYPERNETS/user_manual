\section{Autonomous System Mode}
% \subsection{Protocol Definitions}
% \subsubsection{Protocol v1}
% \subsubsection{Protocol v2}

\subsection{Set Relay n°1 ON at Wake-up}

Before using the studiel-box, we need to set the Yocto-Pictor first 
relay default value as ON. This aims to power-on the rugged-PC at the wake-up time
of the Yocto-Pictor. To achieve this, fire up a terminal and use those commands:
\begin{lstlisting}
cd hypernets_tools/
python -m hypernets.yocto.relay -fpon -n1
\end{lstlisting}

\noindent In order to test if it worked, hold the sleep button (the middle
one) on the Yocto-Pictor until it goes to sleep. Then wake it up by pressing 
the wake-up button (the upper one on the figure \ref{fig:yoctoSleep}). 
The relay 1 should be closed and the associated green LED (i.e. the LED 1 ; on the 
right of the card in the figure \ref{fig:yoctoSleep}) should light.

\begin{figure}[h!]
	\centering
	\cutpic{5pt}{.50\linewidth}{images/yoctopuce_sleepbutton.png}
	\caption{Yocto-Pictor: Three black buttons on the upper-left corner 
			 \& Green Relay Light on the right side of the bottom card}
	\label{fig:yoctoSleep}
\end{figure}
\FloatBarrier

\subsubsection{Enable SSH access}% and Jupyter Notebook (workaround for headless system)}
Closing the Studiel-box will make our system \emph{headless}.
\emph{i.e.} it won't have any keyboard, screen or mouse connected on it. 
As we still need to interact  with it, a workaround solution is to connect to Wi-Fi 
and access it through a SSH, 
\emph{a Secure Shell}. To enable this service, type those two lines in terminal:
\begin{lstlisting}
sudo systemctl enable sshd    # For Manjaro 
sudo systemctl start sshd
sudo systemctl enable ssh    # For Debian 
sudo systemctl start ssh
\end{lstlisting}
Expected output:
\vspace{-10pt}
\begin{lstlisting}
Created symlink /etc/systemd/system/multi-user.target.wants [...]
\end{lstlisting}
Then, try to connect to the Wi-Fi with any laptop and SSH like this 
(you can use \href{https://www.putty.org/}{Putty} on Windows):
\begin{lstlisting}
ssh 'rugged_pc_user_name'@10.42.0.1
\end{lstlisting}

\subsection{Try a full sequence acquisition}
Now edit the \textit{dynamic configuration} file and edit it with the
following values:
\begin{itemize}
	% (see secion :\ref{})
	\item sequence\_file = \textit{path to the sequence of your choice} 
	\item start\_sequence = yes
	\item keep\_pc = on
\end{itemize}

Then, run the command below and see if it works. The output \textit{sequence
directory} should be saved in the folder \textit{DATA}. Its name is prefixed by
\textit{SEQ} and given according to the \textit{current UTC date and time} 
with the following format: \textit{SEQYYYYMMDDTHHMMSS}.
\begin{lstlisting}
./utils/run_service.sh
\end{lstlisting}
%After any modification of your config\_file.ini, you can try 
%this to check if it worked.
If everything is working properly, you can refer to the yoctopuce 
\href{https://www.yoctopuce.com/EN/products/yoctohub-wireless/doc/YHUBWLN1.usermanual.html#CHAP9SEC1}{wake-up scheduler documentation} 
in order to configure the system to wake up autonomously.
In addition, to enable the automatic run of the sequence every time the system
wakes up, run the installation script: 
\begin{lstlisting}
sudo install/04_setup_script_at_boot.sh
\end{lstlisting}

\subsection{Setup Server Communication}

If not already done, generate a ssh key with:
\begin{lstlisting}
ssh -t rsa
\end{lstlisting}

Edit the section \textit{network} of the file \textit{config\_static.ini}, 
create an empty directory in the appropriate folder on the server and run: 
\begin{lstlisting}
sudo install/05_setup_server_communication.sh
\end{lstlisting}

\textit{Example:} if your site name is 



% \FloatBarrier

% relays 2 and 3 and see if you are able to move the pan-tilt and take spectra.
% Annex on calibration offset pan-tilt and geometry.

% \FloatBarrier

% relays 2 and 3 and see if you are able to move the pan-tilt and take spectra.
% Annex on calibration offset pan-tilt and geometry.



% \subsubsection{(optional) Jupyter and Voilà! as a more user friendly access}
% \textcolor{red}{Draft!}: \href{https://jupyter.org}{Jupyter doc} and 
% \href{https://voila.readthedocs.io/en/stable/using.html}{voilà! presentation}.
% \newline
% 
% \begin{lstlisting}
% ./install/DD_install_jupyter.sh
% \end{lstlisting}
% \emph{Note that there is no sudo here.}
% Try then to connect with a webrowser to this address:
% \begin{lstlisting}
% https://10.42.0.1:8888
% \end{lstlisting}
% \textcolor{red}{INSERT SCREENSHOT HERE}
