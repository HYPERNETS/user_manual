\section{Hypernets Tools}

\subsection{Presentation}
\par \textit{hypernets\_tools} is the software developped for the Hypernets System.
It consists in a \textit{Git} repository, it is mainly developped with Bash and
Python (3.7+) languages. It is fully compatible with any UNIX system which
inclues \textit{apt} or \textit{pacman} and a functionnal \textit{systemd}
installation. It is also dependent on the \textit{libhypstar}, the \textit{driver} 
library of the Hypstar.
\textit{hypernets\_tools} is designed in the form of a python package,
containing several folders, grouped by thematic.
Most of the scripts contained within those folders are \textit{callable} 
from a terminal and are \textit{self explanatory} thanks to the \textit{help} 
command. In addition, a \textit{Graphical User Interface} is built on top of those 
scripts, allowing to use some basic functionnalities of the system.
\par The installation of the \textit{hypernets\_tools} is made using the 
\textit{distributed version control system} \textit{Git}. 
If you aren't familiar with this tool, you can refer to the 
\href{https://en.wikipedia.org/wiki/Git}{the wikipedia of Git} or 
its documentation. 

\subsection{Installation}
First clone the repository of the project with:
\vspace{-10pt}
\begin{lstlisting}
git clone https://github.com/hypernets/hypernets_tools
\end{lstlisting}
Check-out the appropriate branch with:
\vspace{-10pt}
\begin{lstlisting}
cd hypernets_tools/
git branch -r            # List all remote branches
git checkout main        # Checkout the branch you want
\end{lstlisting}
You should now be able to use the installation wizard:
\vspace{-10pt}
\begin{lstlisting}
sudo ./install/EE_wizard.sh    
\end{lstlisting}

\noindent  This wizard will prompt you a menu where you can:
\begin{enumerate}
	\item Update hypernets\_tools
	\item Install Dependencies
	\item Download and install YoctoHub
	\item Run Yocto-Pictor auto-config  
	\item install / update libhypstar
	\item Configure Hypstar port
\end{enumerate}

\noindent Once every step is done, you can try to launch the \textit{Graphical User
Interface} (and refer to the section \ref{sec:firstTests} for the first tests) with:
\vspace{-10pt}
\begin{lstlisting}
python -m hypernets.gui
\end{lstlisting}

\subsection{Configuration Files}
All parameters of the \textit{hypernets\_tools} can be found in 
text files called \textit{configuration files}.
Splitted in two different parts, the first one is called
\textit{static} (see annex \ref{annex:staticconfig}) ; and the second one \textit{dynamic} 
that will be synchronized over network allowing the user to perform remote 
setting-up of the system (see annex \ref{annex:dynamicconfig}). You can edit it
using any text editor (e.g. vi, nano, ...). 


\begin{lstlisting}
ls 
config_static.ini config_dynamic.ini
\end{lstlisting}

\subsection{Common Examples of Command}
As a python application, you can \textit{call} most of the modules and
submodules of \textit{hypernets\_tools} using the \textit{python -m
hypernets.module.submodule} syntax.
Moreover, the command "--help" with a package name can provide informations
about how to use the package on the \textit{Command Line Interface}.
Note also that arguments are following the GNU argeument syntax convention : 

\noindent\href{https://www.gnu.org/software/libc/manual/html\_node/Argument-Syntax.html}{https://www.gnu.org/software/libc/manual/html\_node/Argument-Syntax.html}

\subsubsection{Driving the Yocto-Pictor}
\vspace{-16pt}
\begin{lstlisting}
python -m hypernets.yocto.relay --help
\end{lstlisting}
\vspace{-10pt}
\lstinputlisting{logs/yocto_relay_help.txt}

\noindent Closing the relay number one:
\vspace{-10pt}
\begin{lstlisting}
python -m hypernets.yocto.relay --state on --id-relay 1  # long version
python -m hypernets.yocto.relay -son -n1                 # short version
\end{lstlisting}

\subsubsection{Moving the Pan-Tilt}
\vspace{-16pt}
\begin{lstlisting}
python -m hypernets.geometry.pan_tilt --help
\end{lstlisting}
\vspace{-10pt}
\lstinputlisting{logs/pan_tilt_help.txt}
\begin{lstlisting}
python -m hypernets.geometry.pan_tilt --pan 90 --tilt 180 --wait # long
python -m hypernets.geometry.pan_tilt -w -p 90 -t 180            # short
\end{lstlisting}


\subsection{Protocol Files}
In order to process an automatic sequence file (i.e. a multitude of measurement
from diverse geometry), you have to first define a \textit{sequence file}. A
bunch of example of sequence file are given in the hypernets/ressource/sequence\_sample 
folder. Edit then the field \textit{sequence\_file} under the general section
of \textit{config\_dynamic.ini}. Note that the path to your sequence
file can be absolute or relative path (see Annex \ref{annex:dynamicconfig}).
\noindent Prior to any other instruction, the file has to start with this line:
\vspace{-10pt}
\begin{lstlisting}
HypernetsProtocol v2.0
\end{lstlisting}
\vspace{-6pt}
\noindent The general syntax to make a measurement can be given as a
\textit{one line instruction}:
\vspace{-10pt}
\begin{lstlisting}
@[geometry1, #flag1, #flag2] + I.measurement1 + J.measurement2
\end{lstlisting}
\vspace{-6pt}
or like this:
\vspace{-10pt}
\begin{lstlisting}
@[geometry2] 
      + K.measurement3 
	  + L.measurement4
\end{lstlisting}

\noindent Whith \textit{geometry1} and \textit{geometry2} for two geometries;
\textit{I, J, K, L} are the counts (integers, number of measurement) 
associated respectively with \textit{measurement1, measurement2, measurement3
and measurement4};
and \textit{flag1} and \textit{flag2} are two associated conditions for the
geometry1.

\subsubsection{Geometry}
A geometry always starts with the symbol "@" and contains a \textit{pan
} (azimuth) coordinate follow by a pan reference, and a \textit{tilt} (zenith) 
coordinate follow by a tilt reference. Optionally, one or several flags 
(conditionning flag) can be specified in a geometry.

\vspace{10pt}
\noindent 3 types of geometry reference are understandable :
\begin{itemize}
	\item \textit{abs}  : absolute (given as the pan-tilt reference)
	\item \textit{sun}  : sun azimuth and zenith as a reference
	\item \textit{hyper}: north (pan) and nadir (tilt) as a reference 
\end{itemize}

\subsubsection{Measurements}
Measurement can be of 3 different types:
\begin{itemize}
	\item Radiometer: defined as a type (\textit{vnir, swir, both)} and 
		an entrance (\textit{rad, irr, dark}), follow by two integration times
		concatenated by dots.
	\item Picture
	\item Validation
\end{itemize}

\subsubsection{Examples}
\vspace{-10pt}
\begin{lstlisting}
@[0, sun, 0, sun] + 1.picture           # Take a picture of the sun
@[0, hyper, 90, hyper]                  # Point to the horizon north
@[198, abs, hyper] + 20.validation      # Make 20 VM measurements
@[90, sun, 180, hyper] + 6.vnir.irr.0.0 # Make 6 irradiance measure
\end{lstlisting}

\begin{lstlisting}
HypernetsProtocol v2.0

# Flag definitions :
~ #firstLessThanFourSec := $spectra_file1.it_vnir <= 4096
~ #sequenceWasFastEnough := $elapsed_time < 120

@[ 90.0, sun, 180.0, hyper ]
	+ 1.vnir.irr.0.0
# Picture of the Sun only if sequence is < 2 minutes
@[ 0.0, sun, 0.0, sun, #firstLessThanFourSec ]
	+ 1.picture
# Park only if first integration time is less than 4 second :
@[ 0.0, sun, 0.0, hyper, #sequenceWasFastEnough ]
\end{lstlisting}


% Note: don’t forget to set-up the serial port configuration for
% the pan-tilt (see quickstart guide 2.3.3)
